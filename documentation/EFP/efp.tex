%% LyX 1.3 created this file.  For more info, see http://www.lyx.org/.
%% Do not edit unless you really know what you are doing.
\documentclass[english]{article}
\usepackage[T1]{fontenc}
\usepackage[latin1]{inputenc}
\usepackage{geometry}
\geometry{verbose,a4paper,tmargin=2cm,bmargin=2cm,lmargin=2cm,rmargin=2cm}
\usepackage{setspace}

\makeatletter

%%%%%%%%%%%%%%%%%%%%%%%%%%%%%% LyX specific LaTeX commands.
%% Bold symbol macro for standard LaTeX users
\newcommand{\boldsymbol}[1]{\mbox{\boldmath $#1$}}


\usepackage{babel}
\makeatother
\begin{document}

\title{EFP/DFT method for water as solvent}


\author{Aleksey M. Shor}

\maketitle
\begin{onehalfspace}

\section{Overview of the method}
\end{onehalfspace}

\begin{onehalfspace}
The effective fragment method (EFM) belong to the wide family of hybrid
QM/MM methods, but differs from most part of them by rather complex
potential acting between quantum mechanical (QM) part and molecular
mechanical (MM) solvent and MM solvent molecules themselves. The method
was derived in two versions. Originally the first variant of EFM was
implemented within framework of Hartree-Fock (HF) approach \cite{Day 1996}.
Later the variant of EFM for running QM/MM calculations at density
functional (DFT) QM scheme was derived \cite{Adamovic 2003}.

In its DFT variant the effective fragment potential (EFP/DFT) is represented
by set of one-electron potentials that are included to the \emph{ab
initio} electronic Hamiltonian. The EFP/DFT contains three energy
terms:
\end{onehalfspace}

\begin{enumerate}
\begin{onehalfspace}
\item Coulombic interactions between solvent molecules (fragment - fragment)
and fragments with QM solute molecules (QM - fragment). This interaction
also accounts for charge penetration effect;
\item Polarization interaction between fragments and fragments with QM solute
molecules;
\item Repulsion terms acting again between fragments and fragments with
QM solute molecules.\end{onehalfspace}

\end{enumerate}
\begin{onehalfspace}
Electrostatic interactions as the Coulomb, polarization and charge
penetration (or screening) contributions are defined exclusively from
DFT calculations on the single water molecule. The repulsion terms
are determined by a fitting procedure to reproduce the QM potential
of the water dimer \cite{Day 1996,Adamovic 2003}.

As typical for hybrid QM/MM methods the system is divided into two
subsystems: a QM part which can include besides the solute molecules
also some of the solvent molecules, and the fragment part which consists
of the solvent molecules represented as fragment potential described
above. Therefore the total Hamiltonian of the whole system is defined
as\begin{equation}
H_{total}=H_{QM}+V_{ef}\label{eq:1}\end{equation}
and here \begin{equation}
V_{ef}=\sum_{i=1}^{N}\left(\sum_{k=1}^{K}V_{k,i}^{elec}+\sum_{l=1}^{L}V_{l,i}^{pol}+\sum_{m=1}^{M}V_{m,i}^{rep}\right).\label{eq:2}\end{equation}
For $i$th fragment all contributions are over a number ($K$, $L$,
$M$) of expansion points and include electron-fragment, nuclei-fragment
and fragment-fragment interactions. 
\end{onehalfspace}

\begin{onehalfspace}

\section{Electrostatic interactions}
\end{onehalfspace}

\begin{onehalfspace}
The Coulomb potential of the fragment is presented by multicenter,
multipolar expansion of the molecular density. For the water molecule
the expansion is performed upto octopole moments at $K=5$ points
which are \textbf{nuclear centers} and \textbf{bond midpoints}. The
general expression for the electrostatic potential of $i$th fragment
can be written as follows:\begin{equation}
V_{i}^{elec}=\sum_{k}^{K}\left(q_{k}V_{QM}-\sum_{\alpha}^{x,y,z}\mu_{k}^{\alpha}F_{QM}^{\alpha}-\frac{1}{3}\sum_{\alpha,\beta}^{x,y,z}\Theta_{k}^{\alpha\beta}F_{QM}^{\alpha\beta}-\frac{1}{15}\sum_{\alpha,\beta,\gamma}^{x,y,z}\Omega_{k}^{\alpha\beta\gamma}F_{QM}^{\alpha\beta\gamma}\right),\label{eq:3}\end{equation}
where $q$, $\mu$, $\Theta$ and $\Omega$ are charges and dipole,
quadrupole, and octopole moments for the fragments, respectively,
and $V_{QM}$, $F_{QM}^{\alpha}$, $F_{QM}^{\alpha\beta}$, and $F_{QM}^{\alpha\beta\gamma}$
are the QM (or fragment, in case of fragment-fragment interaction)
electrostatic potential, field, field gradient and field Hessian.
\end{onehalfspace}

\begin{onehalfspace}

\section{Charge penetration}
\end{onehalfspace}

\begin{onehalfspace}
The multipolar presentation of the charge density cannot account for
the overlap of the electron charge density between two molecules.
For long distances the multipolar expansion gives good description
of the electrostatic interaction, but needs to be corrected at short
distances when electron charge densities overlap significantly. In
EFP method to correct effect of overlapping a screening function is
used. In EFP/DFT approach only the electronic part of the charge-charge
interaction is screened. The QM Coulomb term is multiplied by a damping
function in the following form:\begin{equation}
\left[1-c_{k}\cdot e^{-a_{k}r_{k,QM}^{2}}\right]q_{k}V_{QM},\label{eq:4}\end{equation}
where $r_{k,QM}$ is the distance between $k$th point of fragment
and QM charge density and $V_{QM}$ is the electrostatic potential
produced by electron density and nuclei. 

The charge-charge interaction between two fragments also is screened.
Here more general expression \cite{Freitag 2000}, which can be applied
to any solvent, is used. For the general case of two different fragments
the energy of their interaction is written as follow:\begin{equation}
E^{ef-ef}=-\frac{q_{1}q_{2}}{\left|r_{12}\right|}\cdot\frac{\left(a_{1}^{2}e^{-a_{2}r_{12}}-a_{2}^{2}e^{-a_{1}r_{12}}\right)}{\left(a_{1}^{2}-a_{2}^{2}\right)}.\label{eq:5}\end{equation}
For the same fragments $\left(a_{1}=a_{2}\right)$ the energy expression
becomes:\begin{equation}
E^{ef-ef}=-\frac{q_{1}q_{2}}{\left|r_{12}\right|}\cdot\left(1+\frac{a_{1}r_{12}}{2}\right)e^{-a_{1}r_{12}}.\label{eq:6}\end{equation}
And for the specific case when one of two charges is not screened
we can write:\begin{equation}
E^{ef-ef}=-\frac{q_{1}q_{2}}{\left|r_{12}\right|}e^{-a_{1}r_{12}}.\label{eq:7}\end{equation}

\end{onehalfspace}

\begin{onehalfspace}

\section{Polarization}
\end{onehalfspace}

\begin{onehalfspace}
The EFP/DFT approach is used the linear polarizability model:\begin{equation}
\vec{\mu}_{i}=\tilde{\alpha}_{i}\vec{F}_{i}.\label{eq:8}\end{equation}
Here, $\vec{\mu}_{i}$ is the induced dipole vector a point $i$,
$\vec{F}_{i}$is the electric field vector at point $i$, and $\tilde{\alpha}_{i}$
is the corresponding tensor of the polarizability. The molecular polarizability
tensor is expressed as a tensor sum of the localized molecular orbital
polarizabilities (LMO) \cite{Boys 1966}, centered at the LMO centroids.
For water molecule, there are five such LMOs: \textbf{oxygen inner
shell, two oxygen lone pairs, two oxygen-hydrogen bonds}.

The total energy of the QM-EFP system due to polarization is\begin{equation}
E_{pol}=E_{int}+E_{ind},\label{eq:9}\end{equation}
$E_{int}$ is the energy of interaction of the induced dipoles with
the electrostatic field produced by the QM part, EFP multipoles and
induced dipoles:\begin{equation}
E_{int}=\sum_{i=1}^{N}\left(-\sum_{l=1}^{L}\vec{\mu}_{l}\vec{F}_{l}^{QM}-\sum_{l=1}^{L}\vec{\mu}_{l}\vec{F}_{l}^{mp}-\frac{1}{2}\sum_{l=1}^{L}\vec{\mu}_{l}\vec{F}_{l}^{\mu}\right),\label{eq:10}\end{equation}
where $\vec{F}_{l}^{QM}$ is the field at point $l$ from the QM part
of the system, $\vec{F}_{l}^{mp}$ and $\vec{F}_{l}^{\mu}$ are the
fields due to the static multipoles and induced dipoles in the other
fragments. The total electric field at point $l$ is given by \begin{equation}
\vec{F}_{l}^{tot}=\vec{F}_{l}^{QM}+\vec{F}_{l}^{mp}+\vec{F}_{l}^{\mu},\label{eq:11}\end{equation}
and Eq. \ref{eq:10} can be rewritten as\begin{equation}
E_{int}=-\sum_{i=1}^{N}\left(\sum_{l=1}^{L}\vec{\mu}_{l}\left[\vec{F}_{l}^{tot}-\frac{1}{2}\vec{F}_{l}^{\mu}\right]\right)=-\sum_{i=1}^{N}\left(\sum_{l=1}^{L}\left[\tilde{\alpha}_{i}\vec{F}_{l}^{tot}\right]\cdot\left[\vec{F}_{l}^{tot}-\frac{1}{2}\vec{F}_{l}^{\mu}\right]\right).\label{eq:12}\end{equation}
$E_{ind}$ is the energy required to induced the dipoles in the fragments.
It was shown \cite{Debye 1945} that\begin{equation}
E_{ind}=\sum_{i=1}^{N}\left(\frac{1}{2}\sum_{l=1}^{L}\vec{\mu}_{l}\vec{F}_{l}^{tot}\right)=\sum_{i=1}^{N}\left(\frac{1}{2}\sum_{l=1}^{L}\vec{\mu}_{l}\left[\vec{F}_{l}^{QM}+\vec{F}_{l}^{mp}+\vec{F}_{l}^{\mu}\right]\right).\label{eq:13}\end{equation}
Therefore the total polarization energy is\begin{equation}
\begin{array}{ccl}
E_{pol} & = & \sum_{i=1}^{N}\left(-\sum_{l=1}^{L}\vec{\mu}_{l}\cdot\left[\vec{F}_{l}^{tot}-\frac{1}{2}\vec{F}_{l}^{\mu}\right]+\frac{1}{2}\sum_{l=1}^{L}\vec{\mu}_{l}\vec{F}_{l}^{tot}\right)\\
 & = & \sum_{i=1}^{N}\left(\frac{1}{2}\sum_{l=1}^{L}\vec{\mu}_{l}\cdot\left[\vec{F}_{l}^{tot}-\vec{F}_{l}^{\mu}\right]\right)\\
 & = & \sum_{i=1}^{N}\left(\frac{1}{2}\sum_{l=1}^{L}\left[\tilde{\alpha}_{i}\vec{F}_{l}^{tot}\right]\cdot\left[\vec{F}_{l}^{QM}+\vec{F}_{l}^{mp}\right]\right).\end{array}\label{eq:14}\end{equation}


The last step is to add the contribution coming from the polarization
energy of fragments to the Kohn-Shem (KS) Hamiltonian in the QM self-consistent-field
(SCF) procedure. The corrected total KS Hamiltonian can be written
as\begin{equation}
H=H_{0}-\frac{1}{2}\sum_{i=1}^{N}\sum_{l=1}^{l}\left(\vec{\mu}_{l}+\vec{\mu}_{l}^{\clubsuit}-\vec{\mu}_{l}^{\prime}\right)\cdot\vec{f}_{l}^{el}.\label{eq:15}\end{equation}
Here, $\vec{\mu}_{l}^{\clubsuit}=\tilde{\alpha}_{i}^{T}\vec{F}_{l}^{tot^{\prime}}$,
$\vec{\mu^{\prime}}_{l}=\tilde{\alpha}_{i}^{T}\vec{F}_{l}^{\mu}$
and $\vec{f}_{l}^{el}$ is the electronic field operator $\left(\vec{F}_{l}^{el}=\left\langle \psi\left|\vec{f}_{l}^{el}\right|\psi\right\rangle \right)$.
$\vec{F}_{l}^{tot^{\prime}}$ here is writen as\begin{equation}
\vec{F}_{l}^{tot^{\prime}}=\vec{F}_{l}^{QM}+\vec{F}_{l}^{mp}+\vec{F}_{l}^{\mu^{\prime}},\label{eq:16}\end{equation}
$\vec{F}_{l}^{\mu^{\prime}}$ is the electic field produced by dipoles
$\vec{\mu}_{l}^{\clubsuit}$.
\end{onehalfspace}

\begin{onehalfspace}

\section{Repulsion terms}
\end{onehalfspace}

\begin{onehalfspace}
For the QM-fragment interaction $V^{rep}$term is represented by linear
combination of two Gaussian functions, expanded at \textbf{the atomic
centers}:\begin{equation}
V_{i}^{rep}=\sum_{m=1}^{M}\left(\sum_{j=1}^{2}c_{mj}e^{-a_{mj}r_{m,QM}^{2}}\right),\label{eq:17}\end{equation}
where $r_{m,QM}$is the distance between $m$th fragment points and
the electronic charge density.

For the fragment-fragment interaction instead of two Gaussian functions,
a single exponential is used expanded at \textbf{the atomic centers
and the center of mass}, in order to better capture the angular dependence
of the charge transfer contribution.
\end{onehalfspace}

\begin{onehalfspace}

\section{Water parameters}
\end{onehalfspace}

\begin{onehalfspace}
The EFP method is used a fixed water model. That is, during geometry
optimization structural parameters of water fragment are kept fixed
at 0.9439 angstrom (O-H bond) and 106.7 degree (H-O-H angle).
\end{onehalfspace}

\begin{onehalfspace}

\section{Geometry optimization}
\end{onehalfspace}

\begin{onehalfspace}
In order to include fragment molecules in the geometry optimization
the forces have to be calculated on them. The energy derivatives are
calculated with respect to each multipole expansion point, each polarizability
point and each repulsion point. But these derivatives cannot be used
directly in geometry optimization due to fixed structure of the fragment.
Therefore we must constrain the points within a fragment to avoid
their moving relatively each other. This is accomplished by combination
of the forces on the fragment to give a three components of a net
force on the fragment and a three components of a net torque around
the fragment center-of-mass. The net force on a fragment is obtained
by summing the forces on each fragment point:\begin{equation}
F_{x}=\sum_{j}\left(-\frac{\partial E}{\partial x_{j}}\right),\label{eq:18}\end{equation}
where $j$ denotes a fragment point.

Calculation of the torque on the fragment is done in general as following:\begin{equation}
\overrightarrow{\tau}=\sum_{j}\left(\overrightarrow{r_{j}}\times\overrightarrow{F_{j}}\right),\label{eq:19}\end{equation}
where $\overrightarrow{r_{j}}$ is a vector from the center of mass
to the point of application of the force, and $\overrightarrow{F_{j}}$
is the force vector for this interaction. To get full torque of the
fragment the contributions coming from dipoles, quadrupoles and octopoles
have to contain additional corrections. For example, for dipole the
completed expression for torque is given by\begin{equation}
\vec{\tau}_{j}=\vec{r}_{j}\times\vec{F}_{j}+\vec{\mu}_{j}\times\vec{\epsilon}_{j}.\label{eq:20}\end{equation}
Here $\vec{\epsilon}_{j}$ is the electrostatic field at the $j$th
position of dipole $\vec{\mu}_{j}$.

The additional term $\vec{\tau}_{ad}$ contributes to the torques
on a quadrupole is written as follow:\begin{equation}
\begin{array}{ccl}
\tau_{ad}^{x} & = & \frac{1}{3}\sum_{\alpha}^{x,y,z}\left(\Theta^{y,\alpha}\epsilon_{\alpha,z}^{\prime}-\Theta^{z,\alpha}\epsilon_{\alpha,y}^{\prime}\right)\\
\tau_{ad}^{y} & = & \frac{1}{3}\sum_{\alpha}^{x,y,z}\left(\Theta^{z,\alpha}\epsilon_{\alpha,x}^{\prime}-\Theta^{x,\alpha}\epsilon_{\alpha,z}^{\prime}\right)\\
\tau_{ad}^{z} & = & \frac{1}{3}\sum_{\alpha}^{x,y,z}\left(\Theta^{x,\alpha}\epsilon_{\alpha,y}^{\prime}-\Theta^{y,\alpha}\epsilon_{\alpha,x}^{\prime}\right)\end{array}.\label{eq:21}\end{equation}
$\epsilon^{\prime}$ is the field gradient at the position of quadrupole
$\Theta$. The correction to the torque on an octupole is given in
\cite{PAPS 1996}.
\end{onehalfspace}

\begin{onehalfspace}

\section{Energy derivatives}
\end{onehalfspace}

\begin{onehalfspace}
Derivation of the gradients of the energy of the QM part is well known.
Also standard procedures are used to derive the gradients of the interaction
energy between the QM part and EFP multypole and repulsive centers
and between these fragment centers themselves. The special attention
has to be paid to derive the gradients of the polarization energy
(Eq.\ref{eq:14}). EFM is used asymmetric anisotropic polarizability
tensors \cite{Adamovic 2003}. For symmetric isotropic polarizability
tensors, the forces can be derived based on a variational treatment
of the polarization energy, which leads to vanishing induced dipole
derivatives \cite{Day 1996}. But for anisotropic polarizability tensor
a variational method cannot lead to the correct expressions for the
forces and torques at the polarizability points, since the derivatives
of the induced dipoles have to be explicitly evaluated. To avoid evaluation
of the induced dipole derivatives in \cite{Li 2006} the forces and
torques acting on polarizability points were derived via a direct
differentiation approach in the form of matrix equations. As formulated
in \cite{Li 2006} the gradients of the polarization energy can be
written as\begin{equation}
g_{\alpha_{i}}=-\frac{1}{2}\left[\left(\vec{F}^{tot^{\prime}}\right)_{\alpha_{i}}^{T}\cdot\vec{\mu}_{i}+\left(\vec{F}^{tot}\right)_{\alpha_{i}}^{T}\cdot\vec{\mu}_{i}^{\clubsuit}\right].\label{eq:22}\end{equation}
Here $\alpha$ are $x$, $y$ and $z$. The Eq. \ref{eq:22} can be
rewritten in more suitable form for implementing:\begin{equation}
g_{\alpha_{i}}=-\frac{1}{2}\left[\left(\vec{F}^{QM}\right)_{\alpha_{i}}^{T}\cdot\left(\vec{\mu}_{i}+\vec{\mu}_{i}^{\clubsuit}\right)+\left(\vec{F}^{mp}\right)_{\alpha_{i}}^{T}\cdot\left(\vec{\mu}_{i}+\vec{\mu}_{i}^{\clubsuit}\right)+\left(\vec{F}^{\mu^{\prime}}\right)_{\alpha_{i}}^{T}\cdot\vec{\mu}_{i}+\left(\vec{F}^{\mu}\right)_{\alpha_{i}}^{T}\cdot\vec{\mu}_{i}^{\clubsuit}\right].\label{eq:23}\end{equation}
Similarly, the torques at polarizability points can be presented:\begin{equation}
\begin{array}{ccl}
\tau_{i} & = & \frac{1}{2}\left(\vec{F}^{tot^{\prime}}\times\vec{\mu}_{i}+\vec{F}^{tot}\times\vec{\mu}_{i}^{\clubsuit}\right)\\
 & = & \frac{1}{2}\left(\vec{F}^{QM}\times\left[\vec{\mu}_{i}+\vec{\mu}_{i}^{\clubsuit}\right]+\vec{F}^{mp}\times\left[\vec{\mu}_{i}+\vec{\mu}_{i}^{\clubsuit}\right]\right.\\
 & + & \left.\vec{F}^{\mu^{\prime}}\times\vec{\mu}_{i}+\vec{F}^{\mu}\times\vec{\mu}_{i}^{\clubsuit}\right).\end{array}\label{eq:23}\end{equation}
 
\end{onehalfspace}

\begin{thebibliography}{5}
\bibitem{Day 1996}P. N. Day, J. H. Jensen, M. S. Gordon, S. P. Webb, W. J. Stevens,
M. Krauss, D. Garmer, H. Basch, D. Cohen, J. Phys. Chem. 105 (1996)
1968.
\bibitem{Adamovic 2003}I. Adamovic, M. A. Freitag, M. S. Gordon, J. Phys. Chem. 118 (2003)
6725.
\bibitem{Freitag 2000}M. A. Freitag, M. S. Gordon, J. H. Jensen, W. J. Stevens, J. Phys.
Chem. 112 (2000) 7300.
\bibitem{Boys 1966}S. F. Boys, in P. O. Lowdin (Eds.), Quantum Science of Atoms, Molecules
Solids, Academ Press, NY, 1966, pp. 253-262. 
\bibitem{Debye 1945}P. Debye, Polar Molecules, Dover, Mineola, NY, 1945, p. 30.
\bibitem{PAPS 1996}See AIP document no. PAPS JCPSA-105-1968-20.
\bibitem{Li 2006}H. Li, H. M. Netzloff, M. S. Gordon, J. Phys. Chem. 125 (2006) 194103.\end{thebibliography}

\end{document}
