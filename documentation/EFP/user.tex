%% LyX 1.3 created this file.  For more info, see http://www.lyx.org/.
%% Do not edit unless you really know what you are doing.
\documentclass[12pt,american]{article}
\usepackage[T1]{fontenc}
\usepackage[iso88595]{inputenc}
\usepackage{geometry}
\geometry{verbose,a4paper,tmargin=2cm,bmargin=2cm,lmargin=2cm,rmargin=2cm}
\usepackage{setspace}
\onehalfspacing

\makeatletter

%%%%%%%%%%%%%%%%%%%%%%%%%%%%%% LyX specific LaTeX commands.
%% Bold symbol macro for standard LaTeX users
\newcommand{\boldsymbol}[1]{\mbox{\boldmath $#1$}}


%%%%%%%%%%%%%%%%%%%%%%%%%%%%%% User specified LaTeX commands.
\usepackage{setspace}


\setlength{\parskip}{\smallskipamount}
\setlength{\parindent}{0pt}

\usepackage{array}



\usepackage{babel}
\makeatother
\begin{document}

\title{The effective fragment potential (EFP) method in \\
 \textsc{ParaGauss}.\\
 (user manual)}


\author{Aleksey M. Shor}

\maketitle

\section{Introduction}

The effective fragment potential (EFP) approach belongs to the large
family of hybrid QM/MM methods realized for studying chemical processes
in solutions. The original version of the EFP method was derived within
the quantum chemical package GAMESS in Mark S. Gordon's theoretical
group of Iowa State University and oriented to model chemical reactions
in water as solvent. There are two versions of the original EFP approach
derived in frameworks of HF and DFT methods. The main feature of the
EFP method is that all parameters used to describe solute-solvent
and solvent-solvent interactions are based on QM reference data. Interesting
persons can be referred to two original papers \cite{Day 96,Adamovic 03}
to get full information on details of EFP method.

The EFP approach in \textsc{ParaGauss} is based on DFT variant of
original method. It means all EFP-DFT parameters were taken from GAMESS
without any modifications. The EFP method uses a {}``so-called''
fixed water model and in this part is similar to simple water models
as TIPn or SPC. Structural parameters of water molecule are 0.9439
\AA(O-H) and 106.7$^{\circ}$(H-O-H). These data are differed slightly
from those presented in the paper \cite{Adamovic 03} but they are
used in GAMESS code and therefore we also use them in our implementation.
TIPn and SPC water models are very simple. They use from 3 to 5 centers
(just point charges) to model electrostatic interaction and only center
to model van der Waals interaction. On the contrary EFP water model
is much complex. Five multipole centers (up to octopoles) are used
to describe electrostatic interactions. Additional five centers are
applied to describe polarization of water molecule. Four centers are
used to mimic repulsion between waters themselves and EFP water and
QM solute. Such complex definition of EFP water model allows to reproduce
very accurately energetic and structural parameters of a system QM
solute + EFP water solvent.


\section{Running EFP calculations}

As the EFP method belongs to QM/MM family to begin EFP calculations
an user have to activate in \textsc{ParaGauss} \textbf{input} file
QM/MM option in namelists \texttt{\textsc{OPERATIONS}} of \texttt{\textsc{TASKS}}.\\
 \texttt{\textsc{\&OPERATIONS}}~\\
 \texttt{\textsc{\ldots{}}}~\\
 \texttt{\textsc{OPERATIONS\_QM\_MM\_NEW = TRUE}}~\\
 \texttt{\textsc{\ldots{}}}~\\
 \texttt{\textsc{/OPERATIONS}}\\
 or\\
 \texttt{\textsc{\&TASKS}}~\\
 \texttt{\textsc{\ldots{}}}~\\
 \texttt{\textsc{QM\_MM\_NEW = TRUE}}~\\
 \texttt{\textsc{\ldots{}}}~\\
 \texttt{\textsc{/TASKS}}\\
 All other options which define calculations of energy, gradients
or geometrical optimization are the same as in regular QM runs.

The second step which has to be done is to include in \textbf{input}
file the namelist \texttt{QMMM} and choose EFP as used QM/MM method.\\
 \texttt{\&QMMM}~\\
 \texttt{METHOD = {}``EFP''}~\\
 \texttt{/QMMM}\\
 The \texttt{QMMM} namelist should be included in \textbf{input} file
after namelist \texttt{OPERATIONS} (or \texttt{TASKS}).

In a usual manner an user have to define QM atoms. \textbf{Here one
has to remember that the current implementation of EFP approach can
be run just without symmetry (symmetry group C1)}.

Just after specifying QM atoms EFP water molecules should be defined.
The first EFP namelist is \texttt{EFP\_DATA}:\\
 \texttt{\&EFP\_DATA}~\\
 \texttt{N\_EFP = 0 ~~~~~~~~~~~~~~(default)}~\\
 \texttt{FIXED\_REGION = {}``NO'' ~~~~(default), {}``EFP'',
{}``QM''~}~\\
 \texttt{N\_DENSITY\_UPDATES = 15 ~(default)~}~\\
 \texttt{ENERGY\_CONV = 0.0000001 (default)~}~\\
 \texttt{/EFP\_DATA}\\
 The most important parameter in that namelist and only that has to
defined explicitly is \texttt{N\_EFP}. It defines number of EFP waters
in the system under study. \\
 The remaining three parameters (\texttt{FIXED\_REGION}, \texttt{N\_DENSITY\_UPDATES}
and \texttt{ENERGY\_CONV}) three possible variants of optimization
of the QM/EFP system. The default task is the full optimization of
the whole QM/EFP system (\texttt{FIXED\_REGION = {}``NO''}). Also
it is possible to perform separate optimization of the QM(\texttt{FIXED\_REGION
= {}``EFP''}) and EFP (\texttt{FIXED\_REGION = {}``QM''}) parts.
To optimize the EFP part keeping the QM cluster fixed the exact frozen
density approach is used. It means to reach the minimum of energy
of just the EFP part optimization a user needs to recalculate periodically
electronic density of the QM cluster. The number of density recalculation
is defined by parameter \texttt{N\_DENSITY\_UPDATES}. The only parameter
that controls the convergence of separate optimization of the EFP
part is \texttt{ENERGY\_CONV}. That parameter defines the minimal
deviation of the total QM/EFP energy after EFP optimization taking
place between electronic density updates. The number of geometry steps
of EFP optimization between QM density updates and convergence parameters
are defined as usual by \textsc{ParaGauss} and Optimizer input files.\\
 Just after the namelist \texttt{EFP\_DATA}, \texttt{N}th namelists
\texttt{EFP\_WATER} have to follow:\\
 \texttt{\&EFP\_WATER}~\\
 \texttt{NAME1 = {}``O1''} or \texttt{{}``H2''}~\\
 \texttt{COOR1 = XXX, YYY, ZZZ,}~\\
 \texttt{NAME2 = {}``H2''} or \texttt{{}``O1''}~\\
 \texttt{COOR2 = XXX, YYY, ZZZ,}~\\
 \texttt{NAME3 = {}``H3''}~\\
 \texttt{COOR3 = XXX, YYY, ZZZ}~\\
 \texttt{/EFP\_WATER}\\
 Here parameters \texttt{NAME1(2,3)} specify names of EFP water atoms.
Parameters \texttt{COOR1(2,3)} define coordinates of water atoms.
Atoms in EFP water have predefined names which cannot be changed.
Oxygen is named \texttt{{}``O1''} and two hydrogen have names \texttt{{}``H2''}
and \texttt{{}``H3''}. The valid order of water atoms within \texttt{EFP\_WATER}
namelist is \texttt{{}``O1''}, \texttt{{}``H2''} and \texttt{{}``H3''}
or \texttt{{}``H2''}, \texttt{{}``O1''} and \texttt{{}``H3''}.
It is not easy to specify positions of EFP water atoms to satisfy
its fixed structure. Fortunately program checks input structure of
EFP water molecules and correct atomic positions to guaranty valid
water structure. Therefore an user can define approximate positions
of EFP water atoms. One also should takes into account that program
changes coordinates of hydrogens leaving oxygen in user defined position.


\section{Geometry optimization}

To perform geometry optimization of hybrid QM/EFP system Cartesian
coordinates are used. To use Cartesian optimization an user should
activate parameter \texttt{CART\_COORDINATES} within namelist \texttt{COORDINATES}
of \textbf{optimizer.input} file.\\
 \texttt{\&COORDINATES}~\\
 \texttt{CART\_COORDINATES = TRUE}~\\
 \texttt{DELOCALIZED\_COORDINATES = FALSE}~\\
 \texttt{ZMAT\_COORDINATES = FALSE}~\\
 \texttt{/COORDINATES}\\
 One also should know choosing EFP method for calculation will automatically
activate Cartesian optimization.

As usual to carry out structural optimization \textbf{gxfile} has
to be used. The format of \textbf{gxfile} for Cartesian optimization
is the same as for optimization in internal coordinates but in the
first case an user is avoided to define Z matrix. Below the example
of \textbf{gxfile} for EFP calculation of water trimmer is presented.

\noindent \texttt{\footnotesize 1.00 ~~~~1.110389253612 ~~~~1.335976091545
~~~~1.503212357197 ~1 1 ~~0 0 0 ~~0 0 0} ~\\
 \texttt{\footnotesize 8.00 ~~~~2.658289496409 ~~~~1.552195249838
~~~~0.534740103514 ~2 2 ~~0 0 0 ~~0 0 0} ~\\
 \texttt{\footnotesize 1.00 ~~~~4.025949957238 ~~~~1.505205277090
~~~~1.756694213035 ~3 3 ~~0 0 0 ~~0 0 0} ~\\
 \texttt{\footnotesize 1.02 ~~~-1.945297856212 ~~~-1.089728128238
~~~~0.671226075878 ~4 4 ~~0 0 0 ~~0 0 0} ~\\
 \texttt{\footnotesize 8.01 ~~~-2.072949963024 ~~~-0.054448293629
~~~~2.118044948249 ~4 4 ~~0 0 0 ~~0 0 0} ~\\
 \texttt{\footnotesize 1.03 ~~~-3.532436421078 ~~~~0.948237221584
~~~~1.903789737485 ~4 4 ~~0 0 0 ~~0 0 0} ~\\
 \texttt{\footnotesize 1.02 ~~~~1.546238534532 ~~~-1.108404235283
~~~-1.741414301468 ~5 5 ~~0 0 0 ~~0 0 0} ~\\
 \texttt{\footnotesize 8.01 ~~~~0.169554461403 ~~~-2.142549670753
~~~-2.206890863366 ~5 5 ~~0 0 0 ~~0 0 0} ~\\
 \texttt{\footnotesize 1.03 ~~~-0.043584534092 ~~~-1.950616952037
~~~-3.967322017812 ~5 5 ~~0 0 0 ~~0 0 0} ~\\
 \texttt{\footnotesize ~-1.0 ~~~~0.000000000000 ~~~~0.000000000000
~~~~0.000000000000 ~0 0 ~~0 0 0 ~~0 0 0 0}{\footnotesize \par{}}{\footnotesize \par}

Here the first three lines are QM water. The last six lines describe
two EFP water molecules. As in \textsc{ParaGauss} \textbf{input} file
in \textbf{gxfile} EFP water atoms have predefined names: 8.01 (oxygen),
1.02 and 1.03 (hydrogens). The same \textbf{gxfile} can be used for
all three variants of optimization of the QM/EFP system. But \textbf{Hessian}
files are not the same (\textbf{just remember that}).


\section{Generation of the second water shell utility \protect \protect \\
 (solute\_in\_water)}

The EFP method is used to study processes in water as the solvent.
Usually the solute and the first shell of the solvent molecules (
water) are treated on QM level. The the first shell of waters can
be prepared by hand as hardly exceed ten molecules with well definite
orientations. But already the second water shell contains some tens
molecules whose orientations in a space are quite random.

To make his work easier one can use utility {}``\textbf{solute\_in\_water}''
to distribute water molecules around the QM cluster which can contains
not only the solute but also waters in the first solvent shell. The
algorithm to distribute water molecules is quite simple. The program
has internally stored bulk of water - 216 water molecules within the
cubic box ($18\times18\times18$ \AA$^{3}$) that actually is the
periodic cell. User puts his QM cluster in the center of the cube
(actually the program does it) and the program chooses just those
water molecules that do not overlap QM cluster (sum of van der Waals
atomic radii is used as criteria of overlapping) and that are closer
to QM cluster than the user defined outer radius of the shell.

An input file for {}``\textbf{solute\_in\_water}'' utility is quite
simple:\\
 \texttt{R=R\_O{[}; N=N\_w; S=}\texttt{\underbar{TIP}} \texttt{(or
EFP); W=}\texttt{\underbar{A}} \texttt{(or M); L=}\texttt{\underbar{C}}
\texttt{(or D){]}}\\
 \texttt{N\_qm}\\
 \texttt{NAME\_1~~~~~X\_1~~~~~Y\_1~~~~~Z\_1}~\\
 \texttt{NAME\_2~~~~~X\_2~~~~~Y\_2~~~~~Z\_2}\\
 \texttt{.........................}~\\
 \texttt{NAME\_N\_qm~~X\_N\_qm~~Y\_N\_qm~~Z\_N\_qm}\\
 The first line here is the list of input parameters separated by
{}``\texttt{;}'' and affecting the number and distribution of solvent
water molecules within the shell:\\
 \texttt{R} - the outer radius of the water shell (\AA); \\
 \texttt{N} - the number of water molecules within the water shell;\\
 \texttt{S} - the solvent type: either {}``\texttt{TIP}'' (default)
or {}``\texttt{EFP}''. In the first case user chooses the box containing
TIP4P waters, otherwise the box will be filled by EFP waters;\\
 \texttt{W} - defines rule to remove water molecules. {}``\texttt{A}''
means that water molecule will be removed if just one its \textbf{A}toms
overlaps the QM cluster (default). On the contrary {}``\texttt{M}''
defines water excluding if all atoms of a \textbf{M}olecule overlap
the QM cluster .\\
\texttt{L -} defines location of the QM cluster within water box.
The default parameter is {}``\texttt{C}'' which means that the QM
cluster will be located in the \textbf{C}enter of the box. To put
the QM cluster in a ran\textbf{D}om position the letter {}``\texttt{D}''
has to be chosen.

Any of parameters \texttt{N}, \texttt{S}, \texttt{W} and \texttt{L}
can be dropped. In this case default values will be applied. Only
parameter that cannot be dropped is the outer shell radius (\texttt{R}).

The second input line defines number of QM atoms (\texttt{N\_qm}).
And the next \texttt{N\_qm} lines define atomic names and Cartesian
coordinates (\AA) of the QM cluster.

To execute the utility user just should write in a terminal window:
\textbf{\textit{/path/solute\_in\_water < input}}.

As result three new files will be created in the current directory:
\texttt{QM\_shell.xyz} - usual XYZ file, \texttt{QM\_shell.inp} -
the part of \textsc{ParaGauss} input file, and \texttt{gx.QM\_shell}
- gxfile.

\begin{thebibliography}{1}
\bibitem{Day 96}Day P. N., Jensen J. H., Gordon M. S., Webb S. P., Stevens W. J.,
Krauss M., Garmer D., Basch H., Cohen D. J Chem Phys, 1996, V 105,
P. 1968-1986. 
\bibitem{Adamovic 03}Adamovic I., Freitag M. A., Gordon M. S. J Chem Phys, 2003, V. 118,
P. 6725-6732. \end{thebibliography}

\end{document}
