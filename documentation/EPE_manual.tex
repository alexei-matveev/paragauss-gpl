%% LyX 1.3 created this file.  For more info, see http://www.lyx.org/.
%% Do not edit unless you really know what you are doing.
%\documentclass[12pt,american]{article}
\documentclass[american]{article}
\usepackage[T1]{fontenc}
\usepackage[iso88595]{inputenc}
\usepackage{geometry}
\geometry{verbose,a4paper,tmargin=2cm,bmargin=1.5cm,lmargin=2cm,rmargin=1.5cm}
\usepackage{array}
\usepackage{amsmath}
\usepackage{setspace}
\usepackage{amssymb}

\makeatletter

%%%%%%%%%%%%%%%%%%%%%%%%%%%%%% LyX specific LaTeX commands.
\newcommand{\noun}[1]{\textsc{#1}}
%% Because html converters don't know tabularnewline
\providecommand{\tabularnewline}{\\}

%%%%%%%%%%%%%%%%%%%%%%%%%%%%%% User specified LaTeX commands.
\usepackage{array}

\usepackage{babel}
\makeatother
\begin{document}

\title{New variant of EPE/covEPE calculations with \noun{ParaGauss}\\
(user manual)}


\author{Shor A. M.}

\maketitle

\section{Introduction}

{\large The current document describes the new user interface designed
to simplify performing EPE and covEPE tasks. That is not the first
attempt to make EPE/covEPE calculations more attractive for users,
especially for those of them who has no experience with this not trivial
process, but as we believe the successful one. Just one parameter
handles now all steps of EPE task. Number of EPE steps were reduced.
Input files looks much simple. Lot of diagnostic messages were added
in program source code to help user clarify his errors and problems.
But even now, users are not saved from to get knowledge on theory
of EPE method\cite{Nasluzov2001,Nasluzov2003} and crystallography
or to exploit additional programs and utilities which are not part
of EPE method. }{\large \par}


\section{\label{Prog}Programs and utilities}

\begin{singlespace}
\textbf{\noun{\large ParaGauss}} {\large - the main used program.
Applied to perform hybrid QM/MM EPE/covEPE calculations with help
of} \textbf{\large EPE} {\large module . It cannot be ignored.}\\
\textbf{\large MolMech} {\large - module of} \textbf{\noun{\large ParaGauss}}\noun{\large .}
{\large It can be used to perform initial periodic optimization of
3D or 2D (slab) unit cells of solid systems chosen as support of adsorbed
molecules. This step is very desirable but not essential. Anyone can
start EPE/covEPE task using experimental unit cell geometry. }\\
\textbf{\large Gulp \cite{Gale2003}} {\large - external program used
for the same goal as} \textbf{\large MolMech}{\large . The main} \textbf{\large Gulp}
{\large advantage is to keep symmetry of an unit cell during geometry
optimization. If saving symmetry is not important using} \textbf{\large MolMech}
{\large module is more preferable. As external program} \textbf{\large Gulp}
{\large has to be modified to save optimized unit cell in format recognizable
by} \textbf{\large EPE} {\large module of} \textbf{\noun{\large ParaGauss}}{\large .
Ask responsible person to provide you a suitable} \textbf{\large Gulp}
{\large version.}\\
\textbf{\large ewald\_new} {\large - the utility applied to generate
the array of point charges (PC) modeling periodic electrostatic field
of a support acting on the embedded quantum mechanical (QM) cluster.}\textbf{\large }\\
\textbf{\large genlat} {\large - generates super-cells from the unit
cell.}\\
\textbf{\large xyz2gx} {\large - converts XYZ file of atomic coordinates
to GX file.}\\
\textbf{\large gx2xyz} {\large and} \textbf{\large gx2xyz\_a} {\large -
converts GX file to XYZ format.}\\
\textbf{\large epe2xyz} {\large - converts coordinates of EPE to XYZ
format.}\\
\textbf{\large get\_epe\_r} {\large - converts GX file to epe.r file.}{\large \par}
\end{singlespace}


\section{Files used in EPE/covEPE calculations}


\subsection{Input files}

\textbf{\large input (\ref{input0})} {\large - the standard} \textbf{\noun{\large ParaGauss}}
{\large input file. Defines all QM and EPE tasks.}\\
\textbf{\large gx.file (\ref{gx0})} {\large - the famous} \textbf{\noun{\large ParaGauss}}
{\large GX file slightly modified to correspond EPE format. Keeps
current geometry of QM cluster.}\\
\textbf{\large epe.input} {\large (\ref{epe0},\ref{epe1})- the input
file used by} \textbf{\large EPE} {\large module and} \textbf{\large ewald\_new}{\large .
Defines parameters of the molecular mechanics environment (EPE) and
the PC array.}{\large \par}


\subsection{Output files}

\textbf{\large output} {\large - the} \textbf{\noun{\large ParaGauss}}
{\large output file. Contains only QM data.}\\
\textbf{\large trace\_output} {\large - as usual demonstrates QM SCF
convergence and also EPE relaxation (optimization) process. }\\
\textbf{\large epe.out} {\large - output of} \textbf{\large EPE} {\large module.
Contains huge number of data.}\\
\textbf{\large ewald.output} {\large - output of} \textbf{\large ewald\_new}
{\large utility.}\\
\textbf{\large cell.xyz}{\large ,} \textbf{\large Ewald\_PC.xyz}{\large ,}
\textbf{\large ew\_pc.xyz} {\large and} \textbf{\large screep.xyz}
{\large - output files of} \textbf{\large ewald\_new} {\large utility.
The unit cell and the PC array coordinates in XYZ format.}{\large \par}


\subsection{\label{add_files}Working files}

\textbf{\large cellvec} {\large - the optimized unit cell vectors
and atomic coordinates (core and shell\cite{Dick1958}) in \AA. Currently
this file can be generated by} \textbf{\large MolMech} {\large module
or} \textbf{\large Gulp} {\large and is used by} \textbf{\large EPE}
{\large module and} \textbf{\large ewald\_new}{\large . The three
first lines are coordinates of optimized unit cell vectors. The rest
are atomic Cartesian coordinates. To provide compatibility between
different description of atoms in} \textbf{\large EPE} {\large module
and programs exploited to optimize the unit cell,} \textbf{\large cellvec}
{\large file contains not atomic names but rather atomic types (the
first column).} \texttt{\large N.00} {\large corresponds the core
positions and} \texttt{\large N.01} {\large describes the shell coordinates.}\\
\textbf{\large epe.r} {\large - contains regular positions of QM cluster
(in a.u.) It is generated automatically during the first EPE task
or can be prepared by hand from initial GX file with help of} \textbf{\large get\_epe\_r}
{\large utility. It is used in QM calculations and EPE relaxations}\\
\textbf{\large epe.pcr} {\large - contains the total charges and the
regular positions of atoms in the EPE IIa region (in a.u.). It is
used in QM calculations and appeared after the first EPE task.}\\
\textbf{\large epe.pcc} {\large - contains the core charges and the
current coordinates of atomic cores in the EPE IIa region (in a.u.).}\\
\textbf{\large epe.pcs} {\large - contains the shell charges and the
current coordinates of atomic shells in the EPE IIa region (in a.u.).
Both files} \textbf{\large epe.pcc} {\large and} \textbf{\large epe.pcs}
{\large initially are generated after the first EPE task and altered
after each EPE relaxation task. They are applied in QM calculations.}\\
\textbf{\large epeinout} {\large -} \textbf{\large }{\large binary
file containing the core and shell coordinates of atoms within the
EPE IIa region and used in calculations of an EPE relaxation (altered
after each EPE relaxation task).}\\
\textbf{\large reg\_reference} {\large \={ } contains the regular
EPE positions (cores and shells) of active centers (in a.u.). The
file is appeared after the first EPE task and essential for EPE relaxation
task. }\\
\textbf{\large epe\_reference} {\large - contains the optimized QM
positions of active centers (in a.u.).}\\
\textbf{\large pgepe\_reference} {\large - binary file containing
the information for calculating the electrostatic field of regular
QM cluster (without defect or adsorbed molecule) acting on the EPE
IIa region.} \textbf{\large epe\_reference} {\large and} \textbf{\large pgepe\_reference}
{\large are appeared after the second EPE task and essential to calculate
EPE relaxation.}\\
\textbf{\large ewald.pcr} {\large - the file is generated by} \textbf{\large ewald\_new}
{\large utility and contains the charges and positions (in a.u.) of
PC array mimicking periodic electrostatic field acting on the QM cluster
(essential for QM calculations).}\\
{\large All files described has to be located in the working directory.}{\large \par}


\section{Description of input files}


\subsection{\label{input0}input}

\begin{singlespace}
\textbf{\noun{\large ParaGauss}} \textbf{\large input} {\large file
serves as usual to perform QM tasks and also handles different EPE
tasks. To perform any EPE tasks you need to define as TRUE either}
\texttt{\large operations\_qm\_epe} {\large (in the namelist} \texttt{\large {}``OPERATIONS'')}{\large \par}
\end{singlespace}

\begin{flushleft}\texttt{\&operations}~\\
\texttt{.......}~\\
\texttt{~~operations\_qm\_epe = t}~\\
\texttt{.......}~\\
\texttt{/operations}\end{flushleft}

\begin{flushleft}{\large or} \texttt{\large qm\_epe} {\large (if you
prefer the namelist} \texttt{\large {}``TASKS'')} {\large parameters }\end{flushleft}{\large \par}

\begin{flushleft}\texttt{\&tasks }~\\
\texttt{.......}~\\
\texttt{~~qm\_epe = t}~\\
\texttt{.......}~\\
\texttt{/tasks }\end{flushleft}

\begin{flushleft}{\large and add in any appropriated place the namelist}
\texttt{\large {}``EPE\_DATA''}{\large .}\end{flushleft}{\large \par}

\begin{flushleft}\texttt{\&epe\_data}~\\
\texttt{~~epe\_task = {}``epe task''}~\\
\texttt{/epe\_data}\end{flushleft}

\begin{flushleft}{\large As there is no default EPE task, the namelist}
\texttt{\large {}``EPE\_DATA''} {\large cannot be dropped. All EPE
tasks are defined by means keywords presented in Table \ref{T1}}%
\begin{table}

\caption{\label{T1}Description of EPE tasks}

\begin{center}\begin{tabular}{|c|c|>{\raggedright}p{7cm}|}
\hline 
EPE task&
keywords&
Description\tabularnewline
\hline
\hline 
1&
\texttt{{}``get\_epe\_ref''}&
EPE relaxation without QM cluster, generation of \textbf{epe.r}, \textbf{epe.pcr},
\textbf{epe.pcc}, \textbf{epe.pcs} and \textbf{reg\_reference} files\tabularnewline
\hline 
2&
\texttt{{}``get\_qm\_ref''}&
Optimization of the reference QM cluster and as result generation
of \textbf{epe\_reference} and \textbf{pgepe\_reference} files\tabularnewline
\hline 
3&
\texttt{{}``qm\_cluster\_opt''}&
Optimization of QM cluster containing defect or adsorbed molecule\tabularnewline
\hline 
4&
\texttt{{}``epe\_relax''}&
EPE relaxation around QM cluster\tabularnewline
\hline 
5&
\texttt{{}``qm\_cluster\_opt epe\_relax''}&
Combination of tasks 3 and 4\tabularnewline
\hline
\end{tabular}\end{center}
\end{table}
{\large .}\end{flushleft}{\large \par}


\subsection{\label{gx0}gx.file}

{\large For EPE calculations GX file is very important. It not only
contains current atomic Cartesian coordinates and system of internal
coordinates but also defines atoms belonged to the reference system
and additional short-range interactions between QM cluster and EPE
and even between cluster atoms. The format of GX file used in EPE
calculations is slightly different from what is used in regular QM
calculations. To carry out EPE tasks additional column has to be included
in GX file from the right side. This additional column defines which
atoms belong to the reference system. Such atoms are marked by} \texttt{\large -1}{\large .
Other atoms (dummy and adsorbed) are designated by} \texttt{\large 0}{\large .
Initial GX file used in the EPE tasks 1 and 2 (Table \ref{T1}) has
to contains only atoms belonged to the reference system and dummy
atoms (if necessary). The first column usually contains atomic numbers
but for EPE calculations atoms that experience additional short-range
interactions has to be designated by the correct name defined in the
used file of force field parameters (\ref{List}). Below you can see
the example of GX file valid for EPE calculations.}{\large \par}

\begin{flushleft}\texttt{99.00 30.678210315263 18.336280866326 23.085034496964
~0 ~1 0 0 0 ~0 ~0 ~0 ~0 }~\\
\texttt{99.00 30.678210315263 18.336280866326 22.085034496964 ~0
~2 1 0 0 ~0 ~0 ~0 ~0 }~\\
\texttt{99.00 30.678210315263 17.336280866326 22.085034496964 ~0
~3 2 1 0 ~0 ~0 ~0 ~0 }~\\
\texttt{~8.03 29.678210315263 17.336280866326 22.085034496964 ~1
~4 3 2 1 ~1 16 31 -1 }~\\
\texttt{14.03 28.008861496248 19.728041090221 23.058341159952 ~2
~5 3 2 1 ~2 17 32 -1 }~\\
\texttt{14.03 30.914861001140 14.694702339671 23.058341348925 ~3
~6 3 2 1 ~3 18 33 -1 }~\\
\texttt{~8.11 28.495257541428 22.192646694773 21.288174845090 ~4
~7 3 2 1 ~4 19 34 -1 }~\\
\texttt{~8.11 25.054639184999 18.884831351325 23.066151968056 ~5
~8 3 2 1 ~5 20 35 -1 }~\\
\texttt{~8.11 28.756422520253 20.426215803030 25.966651800815 ~6
~9 3 2 1 ~6 21 36 -1 }~\\
\texttt{~8.11 33.292470194283 13.883630807839 21.288175034062 ~7
10 3 2 1 ~7 22 37 -1 }~\\
\texttt{~8.11 28.707508829749 12.557875531249 23.066151968056 ~8
11 3 2 1 ~8 23 38 -1 }~\\
\texttt{~8.11 31.893278430468 14.993021805558 25.966651800815 ~9
12 3 2 1 ~9 24 39 -1 }~\\
\texttt{14.11 28.004590713504 25.221359582899 21.113545184315 10 13
3 2 1 10 25 40 -1 }~\\
\texttt{14.11 22.100416873749 19.728041090221 23.058342104816 11 14
3 2 1 11 26 41 -1 }~\\
\texttt{14.11 30.958813402699 21.431196688443 27.875592438825 12 15
3 2 1 12 27 42 -1 }~\\
\texttt{14.11 35.670079009481 11.944344330914 21.113545373287 13 16
3 2 1 13 28 43 -1 }~\\
\texttt{14.11 27.960638500917 ~9.577839172904 23.058342293788 14
17 3 2 1 14 29 44 -1 }~\\
\texttt{14.11 33.864812718618 16.397857748920 27.875592438825 15 18
3 2 1 15 30 45 -1 }~\\
\texttt{-1.00 ~0.000000000000 ~0.000000000000 ~0.000000000000 ~0
~0 0 0 0 ~0 ~0 ~0 ~0}\end{flushleft}


\subsection{\label{epe0}epe.input for EPE module}

{\large New input file for} \textbf{\large EPE} {\large module} \textbf{\large epe.input}
{\large is completely different from previous versions. The new variant
of} \textbf{\large epe.input} {\large file does not handle any EPE
tasks that now are defined in} \textbf{\large ParaGauss} {\large }\textbf{\large input}
{\large (\ref{input0}) and serves to define EPE parameters and structure.
Also a internal structure of new} \textbf{\large epe.input} {\large was
revised. Previous input procedures were based mainly on standard Fortran
namelists operetors. For new} \textbf{\large epe.input} {\large special
input procedure were designed which (as we hope) make EPE input file
more flexible and transparent, allow arbitrary order of input data
and definition comments and empty lines in any suitable place. }\\
{\large Main elements of} \textbf{\large epe.input} {\large are so-called
blocks of data that contain definite set of parameters:}{\large \par}

\begin{flushleft}\texttt{\&block\_name}~\\
\texttt{~~....... }~\\
\texttt{~~....... }~\\
\texttt{~~.......}~\\
\texttt{\&end}\end{flushleft}

\noindent {\large All parameters within blocks can be separated in
two groups having or not predefined (default) data. The last ones
can be skipped in input file if you prefer to use default data. If
all parameters within block of data have been skipped such block also
can be skipped. An order of block of data within} \textbf{\large epe.input}
{\large is absolutely arbitrary but we recommend you to organize input
file more or less logically. }{\large \par}

\noindent {\large Bellow we present description of all block of data
and parameters. Keywords and data in brackets (}\texttt{\large {[}
{]}}{\large ) can be skipped. The slash (}\texttt{\large /}{\large )
defines that only one a keyword from a list can be specified. Underlined
keywords and data are default. }{\large \par}


\subsubsection{{\large Unit cell definition }}

\begin{flushleft}\texttt{\&cell\_vectors}~\\
\texttt{~~x1 y1 z1}~\\
\texttt{~~x2 y2 x2}~\\
\texttt{~~x3 y3 z3}~\\
\texttt{\&end}\end{flushleft}

\noindent \texttt{\large Cell\_vectors} {\large block contains coordinates
of vectors defining the unit cell of system used. Here and hereinafter
all geometrical parameters are in \AA.}{\large \par}

\begin{flushleft}\texttt{\&coordinates}~\\
\texttt{~~n\_atoms N {[}}\texttt{\underbar{new}}\texttt{/old{]}}~\\
\texttt{~~name1 x1 y1 z1 {[}}\texttt{\underbar{f}}\texttt{/t{]}}~\\
\texttt{~~......}~\\
\texttt{~~......}~\\
\texttt{~~nameN xN yN zN {[}}\texttt{\underbar{f}}\texttt{/t{]}}~\\
\texttt{\&end}\end{flushleft}

\noindent \texttt{\large Coordinates} {\large block defines number
and Cartesian coordinates of atoms in the unit cell. (}\textbf{\large Important}{\large :
EPE model is based on core-shell atomic model\cite{Dick1958}, but
in} \texttt{\large Coordinates} {\large block an user has to define
only {}``natural'' atomic coordinates which coincide with either
experimental data or positions of cores) }\\
{\large The first line within} \texttt{\large coordinates} {\large block
has to define number of atoms} \texttt{\large N} {\large in the unit
cell (keyword} \texttt{\large n\_atoms}{\large ). Keywords} \texttt{\large new}
{\large and} \texttt{\large old} {\large are used to define a rule
of calculation of the regular atomic positions. Keyword} \texttt{\large old}
{\large has to be specified if used EPE working files were generated
by previous EPE versions. Next} \texttt{\large N}{\large th lines
present atomic names and coordinates and keywords} \texttt{\large t}
{\large or} \texttt{\large f}{\large . User is not free to define
atomic names. They has to be taken from force field parameters file
used (\ref{List}). Keywords} \texttt{\large t} {\large and} \texttt{\large f}
{\large help user to fix (}\texttt{\large t}{\large ) or not (}\texttt{\large f}{\large )
positions of some atoms during relaxation of EPE.}\\
{\large As mentioned above it is strong recommended for EPE calculations
to use fully optimized unit cell saved in} \textbf{\large cellvec}
{\large file (\ref{add_files}). In this case all data defined by}
\texttt{\large cell\_vector} {\large and} \texttt{\large coordinates}
{\large blocks will be replaced by those taken from} \textbf{\large cellvec}
{\large located in the working directory. User has to remember that
the order of atomic types defined in} \texttt{\large coordinates}
{\large block and} \textbf{\large cellvec} {\large file has to coincide
with each other.}\\
{\large We can suggest variant which significantly simplifies structure
and size of} \textbf{\large epe.input} {\large file. If} \textbf{\large cellvec}
{\large file is intended to be used, instead of two blocks described
above one small block} \texttt{\large atom\_names} {\large can be
applied.}{\large \par}

\begin{flushleft}\texttt{\&atom\_names}~\\
\texttt{~~n\_atoms N {[}}\texttt{\underbar{new}}\texttt{/old{]}}~\\
\texttt{~~n\_atom\_types N1}~\\
\texttt{~~name1}~\\
\texttt{~~......}~\\
\texttt{~~nameN1}~\\
\texttt{\&end}\end{flushleft}

\noindent {\large As in case of} \texttt{\large coordinates} {\large block
the order of lines in the block} \texttt{\large atom\_names} {\large is
not free. The first line is analogous to the first line in} \texttt{\large coordinates}
{\large block and defines number of atoms} \texttt{\large N} {\large in
the unit cell. The second line (keyword} \texttt{\large n\_atom\_types}{\large )
defines number of different atomic types} \texttt{\large N1} {\large which
can be taken from} \textbf{\large cellvec} {\large file. Next} \texttt{\large N1}{\large th
lines has to contain names of atomic types valid for a chosen force
field parameters file (\ref{List}). }\\
{\large Only the case when} \texttt{\large atom\_names} {\large block
cannot be used is if user need to fix positions of some EPE atoms. }{\large \par}


\subsubsection{{\large Definition of force field acting within EPE region}}

{\large In new variant of EPE input file user is saved from defining
parameters of molecular mechanics (MM) force field acting between
atoms in EPE by hand. Now all force field parameters acting in MM
part and between QM cluster and EPE are saved in public available
files. Only action that user has to do is to define name of force
field file in block} \texttt{\large force\_field}{\large .}{\large \par}

\begin{flushleft}\texttt{\&force\_field}~\\
\texttt{~~file\_name}~\\
\texttt{\&end}\end{flushleft}

\noindent {\large Currently three MM force fields were implemented
for EPE calculations. They can be used to model different silicates
and zeolites \cite{Nasluzov2003,Ivanova2005} aluminium and magnesium
oxides. }{\large \par}

\noindent {\large If user want to extend the set of public force field
parameters, he should have his personal copy of a force field file
to modify it. To use personal copy of a force field parameters file
user has to define its location by means global variable} \texttt{\large TTFSLIBS}{\large .}{\large \par}


\subsubsection{{\large Parameters of EPE regions}}

\begin{flushleft}\texttt{\&epe\_parameters }~\\
\texttt{~~{[}radius\_2a\_region} \texttt{\underbar{10.0}}\texttt{{]}
}~\\
\texttt{~~{[}radius\_2b\_region} \texttt{\underbar{22.0}}\texttt{{]}
}~\\
\texttt{~~{[}}\texttt{\underbar{no\_c3\_symm}}\texttt{/c3\_symm{]}
}~\\
\texttt{~~{[}n\_gen\_ions} \texttt{\underbar{6000}}\texttt{{]} }~\\
\texttt{~~{[}n\_primitive\_cell\_trans} \texttt{\underbar{5}} \texttt{}\texttt{\underbar{5}}
\texttt{}\texttt{\underbar{5}}\texttt{{]} }~\\
\texttt{\&end}\end{flushleft}

\noindent \texttt{\large Epe\_parameters} {\large block defines a
size of generated EPE regions\cite{Nasluzov2001}. The most of keywords
used are self explained. Here} \texttt{\large radius\_2a\_region}
{\large defines radius of IIa region. IIb region is defined by keyword}
\texttt{\large radius\_2b\_region}{\large . The maximal number of
generated ions (atoms) is set by} \texttt{\large n\_gen\_ions} {\large keyword.
To generate EPE regions the unit cell has to be multiplied along three
unit cell vectors (two vectors for slab). The keyword} \texttt{\large n\_primitive\_cell\_trans}
{\large is responsible for unit cell multiplication. If the EPE posses
$C_{3}$symmetry (or similar one) keyword} \texttt{\large c3\_symm}
{\large can be specified.}{\large \par}


\subsubsection{{\large Parameters of EPE relaxation (optimization)}}

\begin{flushleft}\texttt{\&optimization }~\\
\texttt{~~{[}iterations} \texttt{\underbar{100}}\texttt{{]} }~\\
\texttt{~~{[}max\_grad} \texttt{\underbar{0.0001}}\texttt{{]} }~\\
\texttt{~~{[}hess\_update} \texttt{\underbar{10}}\texttt{{]} }~\\
\texttt{~~{[}hess\_weight} \texttt{\underbar{1.0}}\texttt{{]}}~\\
\texttt{~~{[}}\texttt{\underbar{no\_print\_grads}}\texttt{/print\_grads{]}
}~\\
\texttt{\&end}\end{flushleft}

\noindent {\large The main aim of EPE module is optimization of EPE
around QM cluster. This task is handled by} \texttt{\large optimization}
{\large blocks. Keyword} \texttt{\large iterations} {\large defines
maximal number of geometry iterations.} \texttt{\large Max\_grad}
{\large sets the main characteristic of convergence \={ } maximal
absolute component of energy gradients. Keyword} \texttt{\large hess\_update}
{\large defines maximum number of iteration between hessian updates. Initial Hessian
(the unit matrix) can be weighted by number that is specified by keyword}
\texttt{\large hess\_weight}{\large . To output final gradients an
user can apply keyword} \texttt{\large print\_grads}{\large .}{\large \par}


\subsubsection{{\large Output of EPE region}}

\begin{flushleft}\texttt{\&xyz\_output}~\\
\texttt{~~{[}}\texttt{\underbar{no\_output\_IIa}}\texttt{/output\_IIa{]}}~\\
\texttt{\&end}\end{flushleft}

\noindent {\large Block} \texttt{\large xyz\_output} {\large serves
only to save in XYZ format regular positions of ions (atoms) of IIa
region (keyword} \texttt{\large output\_IIa}{\large ).}{\large \par}


\subsection{\label{epe1}epe.input for ewald\_new}

\textbf{\large ewald\_new (\ref{Prog})} {\large routine generates}
\textbf{\large ewald.pcr (\ref{add_files})} {\large file which is
very important to correctly estimate electrostatic effect of EPE on
embedded QM cluster. That is why the same unit cell has to be used
for both programs} \textbf{\large EPE} {\large module and} \textbf{\large ewald\_new}{\large .
To guaranty the last condition and make user work easier in the current
EPE user interface} \textbf{\large ewald\_new} {\large utility exploits
the same EPE input file} \textbf{\large epe.input} {\large (\ref{epe0}).
For} \textbf{\large ewald\_new} {\large only the blocks of data} \texttt{\large cell\_vectors}{\large ,}
\texttt{\large coordinates}{\large ,} \texttt{\large atom\_names}
{\large (replacing two first) and} \texttt{\large force\_field} {\large are
important. Other EPE specific blocks are ignored. Also there are two
blocks of data that has to be included in} \textbf{\large epe.input}
{\large to provide a correct work of} \textbf{\large ewald\_new}{\large .}{\large \par}

\begin{flushleft}\texttt{\&pc\_array}~\\
\texttt{~~{[}radius} \texttt{\underbar{15.0}}\texttt{{]}}~\\
\texttt{\&end}\end{flushleft}

\noindent {\large The block} \texttt{\large pc\_array} {\large has
only keyword} \texttt{\large radius} {\large that sets radius of}
\textbf{\large ewald.pcr} {\large array of point charges. Its value
affects the total number of PC.}{\large \par}

\begin{flushleft}\texttt{\&screep\_data}~\\
\texttt{~~{[}center} \texttt{\underbar{0.0}} \texttt{}\texttt{\underbar{0.0}}
\texttt{}\texttt{\underbar{0.0}}\texttt{{]}}~\\
\texttt{~~{[}radius} \texttt{\underbar{5.0}}\texttt{{]}}~\\
\texttt{~~{[}screep\_points} \texttt{\underbar{128}}\texttt{{]}}~\\
\texttt{~~{[}}\texttt{\underbar{c3}}\texttt{/c3v/c4{]}}~\\
\texttt{~~{[}direction} \texttt{\underbar{0.0}} \texttt{}\texttt{\underbar{0.0}}
\texttt{}\texttt{\underbar{0.1}}\texttt{{]}}~\\
\texttt{\&end}\end{flushleft}

\noindent \texttt{\large Screep\_data} {\large block specifies parameters
responsible for generation of SCREEP sphere\cite{Stefanovich1998}.
Here keyword} \texttt{\large center} {\large defines the center of
SCREEP sphere and the whole array of PC that should be located near
the center of QM cluster. Keyword} \texttt{\large radius} {\large sets
the radius of SCREEP sphere. (}\textbf{\large Important}{\large :
the value of radius has to guaranty that QM cluster completely located
within SCREEP sphere) The quality of electrostatic field acting on
QM cluster depends on number of PC in regular cell positions and also
on number of PC distributed on SCREEP sphere surface. The last number
is defined by keyword} \texttt{\large screep\_points}{\large . Valid
values are 128, 512, 2048 and so on. SCREEP sphere can be oriented
accordingly tree types of symmetry ($C_{3}$, $C_{3v}$ and $C_{4}$)
and an direction of the main axes of symmetry is specified by keyword}
\texttt{\large direction}{\large .}\\
{\large As} \textbf{\large ewald\_new} {\large utility is independent
on} \textbf{\noun{\large ParaGauss}} {\large program it cannot use
global variables defined in} \textbf{\large ttfs} {\large script.
Therefore to help} \textbf{\large ewald\_new} {\large to find location
of the force field parameters file user has to define variables} \texttt{\large TTFSLIBS}
{\large by hand.}\\
{\large To estimate quality of work of} \textbf{\large ewald\_new}
{\large utility user should look in the bottom of} \textbf{\large epe.output}
{\large file. There values of exact and approximated periodic electrostatic
potential calculates at positions of QM cluster atoms (}\textbf{\large gx.file}{\large )
can be found. If no} \textbf{\large gx.file} {\large was found electrostatic
potential is calculated at the center of SCREEP sphere.}{\large \par}


\section{EPE calculation. The order of steps}


\subsection{\label{tp}Preliminary tasks}

\begin{enumerate}
\item {\large Optimization of the unit cell, generation of} \textbf{\large cellvec
(\ref{add_files})} {\large file (}\textbf{\large MolMech} {\large module
or} \textbf{\large Gulp}{\large );}{\large \par}
\item {\large Generation of} \textbf{\large gx.file (\ref{add_files})}
{\large - initial GX file of QM cluster without defect or adsorbed
molecules (}\textbf{\large xyz2gx} {\large or by hand);}{\large \par}
\item {\large Preparation} \textbf{\large input} {\large and} \textbf{\large epe.input}
{\large files (\ref{add_files});}{\large \par}
\item {\large Generation of} \textbf{\large ewald.pcr (\ref{add_files})}
{\large file by means} \textbf{\large ewald\_new} {\large utility
(\ref{Prog}).}{\large \par}
\end{enumerate}

\subsection{\label{t1}The first EPE task - optimization of EPE without QM cluster }

{\large To carry out the first task the working directory has to contain
files :}{\large \par}

\begin{enumerate}
\item \textbf{\large input}{\large ,}{\large \par}
\item \textbf{\large epe.input}{\large ,}{\large \par}
\item \textbf{\large gx.file}{\large \par}
\end{enumerate}
{\large In} \textbf{\large input} {\large file: epe\_task=}\texttt{\large ''get\_epe\_ref''}
{\large .}\\
{\large As result files (\ref{add_files})}{\large \par}

\begin{enumerate}
\item \textbf{\large epe.r}{\large ,}{\large \par}
\item \textbf{\large epe.pcr}{\large ,} \textbf{\large epe.pcs}{\large ,}
\textbf{\large epe.pcc}{\large ,}{\large \par}
\item \textbf{\large reg\_reference} {\large (\ref{add_files})}{\large \par}
\item \textbf{\large epeinout}{\large \par}
\end{enumerate}
{\large are appeared in the working directory.}{\large \par}


\subsection{\label{t2}The second EPE task - optimization of initial QM cluster}

{\large The working directory has to contain the files}{\large \par}

\begin{enumerate}
\item \textbf{\large input}{\large ,} \textbf{\large epe.input}{\large ,}
\textbf{\large gx.file,} \textbf{\large ewald.pcr} {\large ,}{\large \par}
\item \textbf{\large epe.r}{\large ,} \textbf{\large epe.pcr}{\large ,}
\textbf{\large epe.pcs}{\large ,} \textbf{\large epe.pcc,}{\large \par}
\item \textbf{\large reg\_reference} {\large (\ref{add_files}),}{\large \par}
\item \textbf{\large epeinout}{\large .}{\large \par}
\end{enumerate}
{\large In} \textbf{\large input} {\large file: epe\_task=}\texttt{\large \char`\"{}get\_qm\_ref\char`\"{}}
{\large .}\\
{\large The result of calculation}{\large \par}

\begin{enumerate}
\item {\large The new optimized} \textbf{\large gx.file}{\large . }{\large \par}
\item \textbf{\large epe\_reference}{\large ,} \textbf{\large pgepe\_reference}
{\large (\ref{add_files}).}{\large \par}
\end{enumerate}

\subsection{\label{t3}The third EPE task - optimization of QM cluster with defect}

{\large The files}{\large \par}

\begin{enumerate}
\item \textbf{\large input}{\large ,} \textbf{\large epe.input}{\large ,}
\textbf{\large gx.file,} \textbf{\large ewald.pcr}{\large ,}{\large \par}
\item \textbf{\large epe.r}{\large ,} \textbf{\large epe.pcr}{\large ,}
\textbf{\large epe.pcs, epe.pcc}{\large ,}{\large \par}
\item \textbf{\large reg\_reference,} {\large }\textbf{\large epe\_reference}{\large ,}
\textbf{\large pgepe\_reference}{\large \par}
\item \textbf{\large epeinout}{\large \par}
\end{enumerate}
{\large has to exist in the working directory before performing the
third EPE task.}\\
{\large In} \textbf{\large input} {\large file: epe\_task=}\texttt{\large \char`\"{}qm\_cluster\_opt\char`\"{}}
{\large .}\\
{\large The result is the new} \textbf{\large gx.file.}{\large \par}


\subsection{\label{t4}The fourth EPE task - EPE relaxation}

{\large The working directory contains the files}{\large \par}

\begin{enumerate}
\item \textbf{\large input}{\large ,} \textbf{\large epe.input}{\large ,}
\textbf{\large gx.file,} \textbf{\large ewald.pcr}{\large ,}{\large \par}
\item \textbf{\large epe.r}{\large ,} \textbf{\large epe.pcr}{\large ,}
\textbf{\large epe.pcs, epe.pcc}{\large ,}{\large \par}
\item \textbf{\large reg\_reference,} {\large }\textbf{\large epe\_reference}{\large ,}
\textbf{\large pgepe\_reference}{\large \par}
\item \textbf{\large epeinout}{\large \par}
\end{enumerate}
{\large In} \textbf{\large input} {\large file: epe\_task=}\texttt{\large \char`\"{}epe\_relax\char`\"{}}~\\
{\large This task results new} \textbf{\large epe.pcs, epe.pcc} {\large and}
\textbf{\large epeinout} {\large files.}\\
{\large Tasks 3 (\ref{t3}) and 4 (\ref{t4}) should be repeated to
reach geometry convergence QM cluster and EPE.}{\large \par}


\subsection{\label{t5}The fifths EPE task - combination of two previous tasks}

{\large The working directory contains the files}{\large \par}

\begin{enumerate}
\item \textbf{\large input}{\large ,} \textbf{\large epe.input}{\large ,}
\textbf{\large gx.file,} \textbf{\large ewald.pcr}{\large ,}{\large \par}
\item \textbf{\large epe.r}{\large ,} \textbf{\large epe.pcr}{\large ,}
\textbf{\large epe.pcs, epe.pcc}{\large ,}{\large \par}
\item \textbf{\large reg\_reference,} {\large }\textbf{\large epe\_reference}{\large ,}
\textbf{\large pgepe\_reference}{\large \par}
\item \textbf{\large epeinout}{\large \par}
\end{enumerate}
{\large In} \textbf{\large input} {\large file: epe\_task=}\texttt{\large \char`\"{}qm\_cluster\_opt
epe\_relax\char`\"{}}{\large }\\
{\large To perform tasks 2-5 (\ref{t2}-\ref{t5}) users has not to
forget to define the rest of parameters within} \textbf{\large input}
{\large file as for geometry optimization task.}{\large \par}


\section{\label{List}List of valid atomic names that has to be used in epe.input
and gx.file}


\subsection{Alumosilicate force field \texttt{(AlumoSilicate-FF.epe)}}

{\large The names of atoms applied in} \textbf{\large epe.input} {\large file:}{\large \par}

\noindent \texttt{\large O1(8.12)} {\large - common oxygen located
between two} \texttt{\large Si1} {\large atoms (}\texttt{\large Si1-O1-Si1}{\large )
,}\\
\texttt{\large O2(8.22)} {\large - oxygen located between} \texttt{\large Si1}
{\large and} \texttt{\large Al1} {\large (}\texttt{\large Si1-O2-Al1}{\large ),
}\\
\texttt{\large O3(8.32)} {\large - oxygen forming bridging OH group
(}\texttt{\large Si2-O3H1-Al1}{\large ), }\\
\texttt{\large O4(8.42)} {\large - oxygen located between common}
\texttt{\large Si1} {\large atom and} \texttt{\large Si2} {\large having
bond with} \texttt{\large O3} {\large (}\texttt{\large Si2-O4-Si1}{\large ),
}\\
\texttt{\large O5(8.52)} {\large - oxygen forming terminal OH group
(}\texttt{\large Si1-O5-H2}{\large ), }\\
\texttt{\large O6(8.62)} {\large - special oxygen located between
two layers of SiO (}\texttt{\large Si1-O6-Si1}{\large ), MCM-41 and
Edingtonite case, }\\
\texttt{\large Si1(14.12)} {\large - common silicon surrounded by}
\texttt{\large O1}{\large ,} \texttt{\large O4,} {\large }\texttt{\large O5}
{\large and} \texttt{\large O6} {\large oxygens, }\\
\texttt{\large Si2(14.22)} {\large - silicon having bond with} \texttt{\large O3}
{\large oxygen, }\\
\texttt{\large Al1(13.12)} {\large - aluminum atom forming T group}
\texttt{\large Al1(O2)$_{3}$O3}{\large , }\\
\texttt{\large H1(1.12)} {\large - hydrogen of bridging OH group,
}\\
\texttt{\large H2(1.22)} {\large - hydrogen of terminal SiOH group.
}\\
{\large The number in parenthesis are reserved atomic names exploited
in QM calculations of EPE tasks. User cannot uses these names to specified
atoms in} \textbf{\large gx.file}{\large .}\\
{\large The next two names has to be used to mark in} \textbf{\large gx.file}
{\large oxygen and silicon pseudoatoms forming boundary between QM
cluster and EPE.}{\large \par}

\noindent \texttt{\large 8.11} {\large - O{*},}\\
\texttt{\large 14.11} {\large - Si{*}.}{\large \par}


\subsection{Alumina force field \texttt{(Al2O3-FF.epe)}}

{\large The names of atoms used in} \textbf{\large epe.input} {\large file:}{\large \par}

\noindent \texttt{\large O1 (8.12)} {\large - MM oxygen atom,}\\
\texttt{\large Al1 (13.12)} {\large - MM aluminium atom.}{\large \par}

\noindent {\large The names of atoms used in} \textbf{\large gx.file}{\large :}\\
\texttt{\large 8.11} {\large \={ } QM oxygen atom,}\\
\texttt{\large 13.11} {\large - QM aluminium atom,}\\
\texttt{\large 13.21} {\large - TIMP aluminium atom.}{\large \par}


\subsection{Magnesia force field \texttt{(MgO-FF.epe)}}

{\large The names of atoms used in} \textbf{\large epe.input} {\large file:}{\large \par}

\noindent \texttt{\large O1 (8.12)} {\large - MM oxygen atom,}\\
\texttt{\large Mg1 (12.12)} {\large - MM magnesium atom.}{\large \par}

\noindent {\large The names of atoms used in} \textbf{\large gx.file}{\large :}\\
\texttt{\large 8.11} {\large \={ } QM oxygen atom,}\\
\texttt{\large 12.11} {\large - QM madnesium atom,}\\
\texttt{\large 12.21} {\large - TIMP magnesium atom.}{\large \par}

\begin{thebibliography}{1}
\bibitem{Nasluzov2001}Nasluzov, V. A.; Rivanenkov, V. V.; Gordienko, A. B.; Neyman, K. M.;
Birkenheuer, U.; R\"{o}sch, N. \emph{J. Chem. Phys}. \textbf{2001},
\emph{115}, 8157.
\bibitem{Nasluzov2003}Nasluzov, V. A.; Ivanova, E. A.; Shor, A. M.; Vayssilov, G. N.; Birkenheuer,
U.; R\"{o}sch, N. \emph{J. Phys. Chem. B} \textbf{2003}, \emph{107},
2228.
\bibitem{Gale2003}Gale, J. D.; Rohl, A. L. \emph{Mol. Simul.} \textbf{2003}, \emph{29},
291.
\bibitem{Dick1958}Dick, B. G.; Overhauser, A. W. \emph{Phys. Rev.} \textbf{1958}, \emph{112},
90.
\bibitem{Ivanova2005}Ivanova-Shor, E. A.; Shor, A. M.; Nasluzov, V. A.;Vayssilov, G. N.;
R\"{o}sch, N. \emph{J. Chem. Theory Comp}. \textbf{2005}, \emph{1},
459.
\bibitem{Stefanovich1998}Stefanovich, E. V.; Truong, T. N. \emph{J. Phys. Chem. B} \textbf{1998},
\emph{102}, 3018.
\end{thebibliography}

\end{document}
