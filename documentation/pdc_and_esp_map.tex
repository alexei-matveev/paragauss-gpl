%% LyX 1.3 created this file.  For more info, see http://www.lyx.org/.
%% Do not edit unless you really know what you are doing.
\documentclass[american]{article}
\usepackage[T1]{fontenc}
%\usepackage[iso88595]{inputenc}
\usepackage{geometry}
\geometry{verbose,a4paper,tmargin=2cm,bmargin=2cm,lmargin=2cm,rmargin=2cm}
\usepackage{array}

\makeatletter

%%%%%%%%%%%%%%%%%%%%%%%%%%%%%% LyX specific LaTeX commands.
\newcommand{\noun}[1]{\textsc{#1}}
%% Bold symbol macro for standard LaTeX users
\providecommand{\boldsymbol}[1]{\mbox{\boldmath $#1$}}

%% Because html converters don't know tabularnewline
\providecommand{\tabularnewline}{\\}

%%%%%%%%%%%%%%%%%%%%%%%%%%%%%% User specified LaTeX commands.
\usepackage{array}

\usepackage{babel}
\makeatother
\begin{document}

\title{Potential Derived Charges (PDC) \\
 and \\
 ElectroStatic Potential (ESP) maps}


\author{Shor A. M.}

\maketitle

\section{Introduction}

{\large ~~~The electrostatic field produced by molecules, clusters
and largest condense systems plays an important role in their interaction
with the outside world. In such applications as the molecular simulation
(MD), the energy minimization and the hybrid quantum mechanical -
molecular mechanical (QM/MM) methods there is a crucial necessity
to model the electrostatic interaction as most as accurately. In the
most part of studies simple atom-centered monopole models are used.
The straightforward way to compute the charges is to fit the QM molecular
electrostatic potential calculated at points in space surrounding
the molecule. It has been shown this approach yields potential derived
atomic charges which reproduce such molecular properties as the dipole
and quadrupole moments with reasonable accuracy. }{\large \par}

{\large Some methods have been suggested in the past\cite{Singh,Chirlian,Breneman,Basler,Spackman}.
The methods reported in all mantioned papers distinguish one from
the others only in generating the space point grid. In \cite{Singh}
and \cite{Basler} the molecular surface algorithm of Connolly\cite{Connolly}
was used to build a set of points on several shells surrounding the
molecule. The papers \cite{Chirlian,Spackman} describe the method
based on atom-centered spheres. And at last in \cite{Breneman} the
procedure which uses regular grids of point was offered (CHELP procedure).
The methods developed in \cite{Singh,Chirlian,Breneman,Basler} are
very close to each other and very often called as ESP method.}{\large \par}

{\large Our implementation is based on atom-centered interlocking
spheres and also can be referred to ESP methods \cite{Basler}. The
space grid is calculated on the set of molecular surfaces formed by
van der Waals overlaping spheres centered on each atoms in the molecule
(cluster). The scheme of grid calculation is chosen to result in the
same symmetry of space points as the molecule has itself. Values of
point charges centered at atomic position in the molecule are computed
to fit the QM electrostatic potential calculated on space grid. Two
constraints can be applied. The standard constrain is the sum of calculated
charges has to be equal the total charge of the molecule. The second
additional constraint is the demand to reproduce the QM dipole moment
of the considered molecule. The linear least squares fit procedure
suggested in \cite{Chirlian} was used. The calculated ESP $\hat{V_{i}}$
is given by \begin{equation}
\hat{V_{i}}=\sum_{j}\frac{q_{j}}{r_{ij}}\label{eq:1}\end{equation}
 so the function $\chi_{esp}^{2}$ to be minimized is defined as\begin{equation}
\chi_{esp}^{2}=\sum_{i}\left(V_{i}-\hat{V}_{i}\right)^{2}\label{eq:2}\end{equation}
 At the minimum\begin{equation}
\frac{\partial\left(\chi_{esp}^{2}\right)}{\partial q_{j}}=0\label{eq:3}\end{equation}
 where \begin{equation}
\frac{\partial\left(\chi_{esp}^{2}\right)}{\partial q_{j}}=-2\sum_{i}\frac{V_{i}-\hat{V}_{i}}{r_{ij}}=0\label{eq:4}\end{equation}
 forms a system of linear equations which can by solved in matrix
form. Both constraints are imposed via Lagrange multipliers:\begin{equation}
\sum_{i}q_{i}=Q_{total}\label{eq:5}\end{equation}
 and\begin{equation}
\sum_{i}q_{i}\cdot r_{i}=d_{total}\label{eq:6}\end{equation}
} 

{\large It is well known that in a standard ESP procedure where a
quality of fit to the ESP is only criteria, the charges on atoms located
in the molecule behind outer atoms ({}``buried'' or {}``hidden'')
can deviate in wide range and even have unphysical values. This in
general because space points at which the QM ESP is evaluated are
relatively far away from and therefore weaker dependent on values
of {}``hidden'' atomic charges. This feature of a standard ESP model
results very often in poor transferability of derived charges between
different conformers of the same chemical fragments. The most power
and popular method that considerably reduces the problem of {}``hidden''
atoms is the RESP method\cite{Bayly}. It bases on introducing restraints
in form of a penalty function into the least-squares fitting procedure.
With including a penalty function to the charge fitting procedure,
an additional term has tobe added to $\chi_{esp}^{2}$function so
the new function to be minimized is now\begin{equation}
\chi^{2}=\chi_{esp}^{2}+\chi_{rstr}^{2}\label{eq:7}\end{equation}
 and the least squares minimum is now defined as\begin{equation}
\frac{\partial\left(\chi^{2}\right)}{\partial q_{j}}=\frac{\partial\left(\chi_{esp}^{2}\right)}{\partial q_{j}}+\frac{\partial\left(\chi_{rstr}^{2}\right)}{\partial q_{j}}=0\label{eq:8}\end{equation}
} 

{\large Different types of a penalty function can be chosen, but more
stable results are produced in case of a hyperbolic form of penalty
function:\begin{equation}
\chi_{rstr}^{2}=a\sum_{j}\left(\left(q_{j}^{2}+b^{2}\right)^{1/2}-b\right)\label{eq:9}\end{equation}
 where $a$is a scale factor which define the asymptotic limits of
the strength of the restraint and $b$determines the {}``tightness''
of the hyperbola around its minimum. Now the second term in equation
\ref{eq:8} is \begin{equation}
\frac{\partial\left(\chi_{rstr}^{2}\right)}{\partial q_{j}}=a\cdot q_{j}\left(q_{j}^{2}+b^{2}\right)^{-1/2}\label{eq:10}\end{equation}
} 

{\large Modification of the matrix equation\begin{equation}
\mathbf{Aq=B}\label{eq:11}\end{equation}
 which is formed from the system of equations \ref{eq:4} is straightforward.
The off-diaganal elements of $\mathbf{A}$ are the same\begin{equation}
A_{jk}=\sum_{i}\frac{1}{r_{ij}r_{ik}}\label{eq:12}\end{equation}
 but the diagonal elements are now given by \begin{equation}
A_{jj}=\sum_{i}\frac{1}{r_{ij}^{2}}+\frac{\partial\left(\chi_{rstr}^{2}\right)}{\partial q_{j}}\label{eq:13}\end{equation}
} 


\section{User guide}

{\large ~~~The implemented procedure allows to perform calculations
of potential derived charges (PDC) and electrostatic potential (ESP)
2-D maps .\par{}}{\large \par}


\subsection{Starting Electrostatic potential based tasks}

{\large First of all to initiate the calculation of ESP anyone have
to specify either TASK variable into namelist TASKS}{\large \par}

\noindent \bigskip{} \texttt{\noun{\large \&TASK}}~\\
 \texttt{\noun{\large TASK = ''POTENTIAL''}}~\\
 \texttt{\noun{\large . . . . . .}}~\\
 \texttt{\noun{\large /TASK}} \bigskip{}

\noindent {\large or switch on OPERATIONS\_POTENTIAL parameter into
OPERATIONS namelist}{\large \par}

\noindent \bigskip{} \texttt{\large \&OPERATIONS}~\\
 \texttt{\large . . . . . .}~\\
 \texttt{\large OPERATIONS\_POTENTIAL = TRUE}~\\
 \texttt{\large . . . . . .}~\\
 \texttt{\large /OPERATIONS} \bigskip{}

{\large ESP is calculated in {}``single point'' regime, but there
is not necessity to do all required settings by hands. Everything
will be done automatically.\par{}}{\large \par}


\subsection{Namelist {}``POTENTIAL''}

{\large The namelist which handles calculations of PDC and ESP 2-D
map has name} \texttt{\textbf{\large POTENTIAL}}{\large . It can be
dropped if anyone computes PDC with default options. But if it is
necessary this namelist can be included in any place of} \noun{\large ParaGauss}
{\large INPUT file after atomic coordinates specified. }{\large \par}

{\large The main parameter of} \texttt{\textbf{\large POTENTIAL}}
{\large namelist is} \texttt{\textbf{\large POTENTIAL\_TASK}}{\large .
It defines the type of calculations. Currently it is allowed} \texttt{\textbf{\large POTENTIAL\_TASK}}
{\large to have only two values {}``PDC'' or {}``ESP\_MAP''. Depending
on the calculation type chosen the different set of namelist parameters
has to be used.}{\large \par}

\texttt{\textbf{\large USE\_SAVED\_DENSMATRIX}} {\large parameter
allows to use the previously saved density matrix to calculate ESP.
It can dramatically shorten the time of a job (the default value is
FALSE). To save the density matrix anyone should specifies the parameter}
\texttt{\textbf{\large SAVE\_DENSMATRIX}} {\large in namelist RECOVER\_OPTIONS} \bigskip{}

\noindent \texttt{\large \&RECOVER\_OPTIONS}~\\
 \texttt{\large . . . . . .}~\\
 \texttt{\large SAVE\_DENSMATRIX = TRUE}~\\
 \texttt{\large /RECOVER\_OPTIONS}{\large \par}


\subsubsection{Calculation of Potential Derived Charges - Task {}``PDC'' }

{\large To carry out PDC calculations the namelist POTENTIAL contains
the following list of input parameters :}{\large \par}

\noindent \bigskip{} \texttt{\large \&POTENTIAL}~\\
 \texttt{\large POTENTIAL\_TASK = {}``PDC''}~\\
 \texttt{\large N\_SHELLS = 4}~\\
 \texttt{\large VDW\_INCREMENTS = 1.0, 1.2, 1.4, 1.6,}~\\
 \texttt{\large N\_FIXED\_ATOMS = 0}~\\
 \texttt{\large FIXED\_ATOMS = 0, 0, 0, ...}~\\
 \texttt{\large FIXED\_CHARGES = 0.0, 0.0, ...}{\large \par}

\noindent \texttt{\large PDC\_CONSTRAINT = {}``CHARGE''}~\\
 \texttt{\large PDC\_RESTRAINT = {}``NO''}~\\
 \texttt{\large RESP\_A = 0.0005}~\\
 \texttt{\large RESP\_B = 0.1}~\\
 \texttt{\large N\_SYMM\_GROUPS = 0}~\\
 \texttt{\large N\_SYMM\_ATOMS = 0, 0, ...}~\\
 \texttt{\large SYMM\_ATOMS = 0, 0, 0, ...}{\large \par}

\noindent \texttt{\large PDC\_OUTPUT = {}``SHORT''}~\\
 \texttt{\large USE\_SAVED\_DENSMATRIX = FALSE}~\\
 \texttt{\large /POTENTIAL} \bigskip{}

\texttt{\textbf{\large N\_SHELL}} {\large parameter defines the number
of shells surrounding the molecule to calculate space grid (}\texttt{\textbf{\large 1}}{\large $\leq$}\texttt{\textbf{\large N\_SHELL}}{\large $\geq$}\texttt{\textbf{\large 8}}{\large ,
the default value is} \texttt{\textbf{\large 4}}{\large ).}{\large \par}

\texttt{\textbf{\large VDW\_INCREMENTS}} {\large specifies the multiplies
of the atomic van der Waals radii for each molecular shell. The number
of multiplies has to be equal} \texttt{\textbf{\large N\_SHELL}} {\large (
the default values are} \texttt{\textbf{\large 1.0}}{\large ,} \texttt{\textbf{\large 1.2}}{\large ,}
\texttt{\textbf{\large 1.4}} {\large and} \texttt{\textbf{\large 1.6}}{\large ).
The current implementation of PDC bases on the van der Waals radii
used in UFF force field\cite{Rappe} and derived for 103 first elements
of the periodic table. }{\large \par}

\texttt{\textbf{\large PDC\_CONSTRAINT}} {\large specifies the constraint
imposed on calculation of PDC. This parameter has two keywords {}``CHARGE''
(the default value) and {}``DIPOLE''. The first keyword means that
the sum of charges calculated has to be equal total charge of molecule
under study. The second one includes QM dipole moment to the fitting
procedure. In this case} \noun{\large ParaGauss} {\large automatically
compute molecular dipole moment. \par{}}{\large \par}

{\large Three next options allow to exclude charges on certain atoms
from fitting process by fixing their values. }{\large \par}

\texttt{\textbf{\large N\_FIXED\_ATOMS}} {\large determines the total
number of atoms carring the fixed charges (the default value is} \texttt{\textbf{\large 0}}{\large ). }{\large \par}

\texttt{\textbf{\large FIXED\_ATOMS}} {\large contains the list of
the unique (from INPUT file) numbers of {}``fixed'' atoms. }{\large \par}

\texttt{\textbf{\large FIXED\_CHARGES}} {\large parameter defines
the values of the fixed charges. The charges has to be specified in
accordance with the order of atomic numbers in the} \texttt{\textbf{\large FIXED\_ATOMS}}
{\large list. \par{}}{\large \par}

{\large Besides two constraints also RESP restraint\cite{Bayly} can
be applied. The restraint applied is a hyperbolic penalty function
(see equation \ref{eq:9}). During restraint calculation forced charge
symmetrization can be imposed on chosen chemically equivalent atoms. }{\large \par}

\texttt{\textbf{\large PDC\_RESTRAINT}} {\large parameter is used
to switch on restraint and symmetrization computing. It has three
keywords : {}``NO'' (default value), {}``RESP'' (simple restraint
conditions) and {}``RESP+SYMM'' (restraint and forced symmetrization). }{\large \par}

{\large Two parameters} \texttt{\textbf{\large RESP\_A}} {\large and}
\texttt{\textbf{\large RESP\_B}} {\large define the hyperbolic penalty
function (}\emph{\large a} {\large and} \emph{\large b} {\large in
equation \ref{eq:9}). The default values are 0.0005 ({}``weak restraint'')
and 0.1 for} \texttt{\textbf{\large RESP\_A}} {\large and} \texttt{\textbf{\large RESP\_B}}
{\large correspondingly. }{\large \par}

{\large A forced symmetrization can by applied to different atomic
group simultaneously. There are three parameters which determine equivalenced
atoms. }{\large \par}

\texttt{\textbf{\large N\_SYMM\_GROUPS}} {\large defines the number
of different atomic equivalenced groups. }{\large \par}

\texttt{\textbf{\large N\_SYMM\_ATOMS}} {\large contains the number
of equivalenced atoms into each group. }{\large \par}

\texttt{\textbf{\large SYMM\_ATOMS}} {\large defines the list of unique
numbers of atoms to be equivalenced into each group.} \medskip{}

\textit{\small Example of definition of forced symmetrization:}~\\
 \texttt{\small \&POTENTIAL}~\\
 \texttt{\small ... ... ... ...}~\\
 \texttt{\small PDC\_RESTRAINT = {}``RESP+SYMM''}~\\
 \texttt{\small N\_SYMM\_GROUP = 2~~~~~~~~~~~~~~~~~~~~~2
atomic group to be equivalenced}~\\
 \texttt{\small N\_SYMM\_ATOMS = 2,3~~~~~~~~~~~~~~~~~~~2
atoms in the first group and 3 in the second one}~\\
 \texttt{\small SYMM\_ATOMS = 3,4,7,8,9~~~~~~~~~~~~~~~atoms
3 and 4 consist the first group, atoms 7, 8 and 9 are in the second
one}~\\
 \texttt{\small ... ... ... ...}~\\
 \texttt{\small /POTENTIAL} \medskip{}

{\large It is recomended to apply the two step procedure to equivalence
atomic charges. The first step consists in week restraint procedure.
The second one is the strong (}\texttt{\textbf{\large RESP\_A}} {\large =
0.001) restraint and symmetrization procedure. On the second step
some previously defined atoms carry fixed charges calculated during
the first step (details can be found in \cite{Bayly}).}{\large \par}

\texttt{\textbf{\large PDC\_OUTPUT}} {\large defines the of output
information. Three keyword can be used :{}``SHORT'' (the default
value), {}``MIDDLE'' and {}``LONG''.}{\large \par}


\subsubsection{Calculation of 2-D Electrostatic potential map - Task {}``ESP\_MAP''}

{\large The second task which can be done is a calculation of ESP
map. In this case namelist POTENTIAL contains other list of parameters:} \bigskip{}

\noindent \texttt{\large \&POTENTIAL}~\\
 \texttt{\large POTENTIAL\_TASK = {}``ESP\_MAP''}~\\
 \texttt{\large V\_ELECTRONIC = TRUE}~\\
 \texttt{\large V\_NUCLEAR = TRUE}~\\
 \texttt{\large V\_PC = FALSE}~\\
 \texttt{\large FIRST\_POINT = 0.0, 0.0, 0.0}~\\
 \texttt{\large SECOND\_POINT = 1.0, 0.0, 0.0}~\\
 \texttt{\large THIRD\_POINT = 0.0, 1.0, 0.0}~\\
 \texttt{\large XLIMITS = 0.0, 1.0}~\\
 \texttt{\large YLIMITS = 0.0, 1.0}~\\
 \texttt{\large XGRID = 20}~\\
 \texttt{\large YGRID = 20}~\\
 \texttt{\large OUTPUT\_UNITS = {}``a.u.''}~\\
 \texttt{\large OUTPUT\_FORMAT = {}``GNUPLOT''}~\\
 \texttt{\large USE\_SAVED\_DENSMATRIX = FALSE}~\\
 \texttt{\large /POTENTIAL} \bigskip{}

\texttt{\textbf{\large V\_ELECTRONIC}}{\large ,} \texttt{\textbf{\large V\_NUCLEAR}}
{\large and} \texttt{\textbf{\large V\_PC}} {\large parameters define
what part of ESP should be calculated. Here} \texttt{\textbf{\large V\_PC}}
{\large option handles computing of ESP due to point charges, if specified. }{\large \par}

\texttt{\textbf{\large FIRST\_POINT}}{\large ,} \texttt{\textbf{\large SECOND\_POINT}}
{\large and} \texttt{\textbf{\large THIRD\_POINT}} {\large parameters
are three space points defining the plane where ESP to be calculated.
Simultaneously these points determine the 2-D Cartesian coordinate
system connected on the plane.} \texttt{\textbf{\large FIRST\_POINT}}
{\large is the center of coordinates, X axis is directed from} \texttt{\textbf{\large FIRST\_POINT}}
{\large to} \texttt{\textbf{\large SECOND\_POINT}}{\large . Y axis
is calculated automatically. }{\large \par}

\texttt{\textbf{\large XLIMITS}} {\large and} \texttt{\textbf{\large YLIMITS}}
{\large define the limits along X and Y axes correspondingly. }{\large \par}

\texttt{\textbf{\large XGRID}} {\large and} \texttt{\textbf{\large YGRID}}
{\large parameters define the point grid on the chosen chosen. The
total number of points are equal} \texttt{\textbf{\large XGRID}}{\large $\times$}\texttt{\textbf{\large YGRID}}
{\large (the default values is 50 for both parameters). A quality
of ESP map directly depends on a grid magnitude. }{\large \par}

\texttt{\textbf{\large OUTPUT\_UNITS}} {\large specified the units
of ESP calculated and the plane coordinate system. There are two keywords
{}``a.u.'' - atomic units (the default) and {}``eV-a'' (electron
volt and angstrom). }{\large \par}

{\large To visualize ESP map anyone needs to use graphical program.
The parameter} \texttt{\textbf{\large OUTPUT\_FORMAT}} {\large allows
to save final data in three formats:}{\large \par}

\vspace{0.3cm}

\begin{center}\begin{tabular}{|c|p{2cm}|p{3cm}|}
\hline 
{\large OUTPUT\_FORMAT}&
 {\large output files}&
 {\large graphical program}\tabularnewline
\hline
{\large {}``GNUPLOT'' (the default)}&
 {\large V.gpl}&
 {\large gnuplot}\tabularnewline
\hline
{\large {}``SCILAB''}&
 {\large Vpoints\par{}}Xpoints Ypoints\par{}xygrid\par{}Cluster&
scilab\tabularnewline
\hline
{}``WORKSHEET''&
V.txt&
any program under Windows \tabularnewline
\hline
\end{tabular}\end{center}

\vspace{0.3cm}

\begin{thebibliography}{1}
\bibitem{Singh}U. C. Singh, P. A. Kollman, J. Comp. Chem., V.5, N.2, P.129-145, 1984 
\bibitem{Chirlian}L. E. Chirlian, M. M. Francl, J. Comp. Chem., V.8, N.6, P.894-905,
1987 
\bibitem{Breneman}C. M. Breneman, K. B. Wiberg, J. Comp. Chem., V.11, N.3, P.361-373,
1990 
\bibitem{Basler}B. H. Besler, K. M. Merz, P. A. Kollman, J. Comp. Chem., V.11, N.4,
P.431-439, 1990 
\bibitem{Spackman}M. A. Spackman, J. Comp. Chem., V.17, N.1, P.1-18, 1996 
\bibitem{Connolly}M. Connolly, QCPE Program 429, 1982 
\bibitem{Rappe}A. K. Rappe, C. J. Casewit, K. S. Colwell, W. A. Goddard III, W. M.
Skiff. J. Am. Chem. Soc., V.114, P.10024, 1992 
\bibitem{Bayly}C. I. Bayly, P. Cieplak, W. D. Cornell, P. A. Kollman, J. Pys. Chem.,
V.97, P. 10269, 1993
\end{thebibliography}

\end{document}
